% I use scrbook as document class for the following reasons:
% I find that KOMA script in general is a more advanced and flexible
% approach for latex documents than the traditional doc classes. It has way
% more options to format the document and I will thus always use KOMA
% script. The book class is used because I use chapters and might want to
% have empty pages before the beginning of a chapter. This is possible with
% the other classes but default with the scrbook one.
%
% The options are set separately and commented below.
\documentclass{scrbook}

% I use Din A5 as the document format. 
\KOMAoptions{paper = a5}

% The font size will be 12pt.   
\KOMAoptions{fontsize = 10pt}

% Since the output will be a book, I set the twoside option to true.
\KOMAoptions{twoside = true}

% titlepage as separated page
\KOMAoptions{titlepage = true}

% write size of file to output. useful for different output formats
\KOMAoptions{pagesize}

% The BCOR value is important for the binding of the book. It corrects the
% margin on the inner edge of the page that can not be seen after the book
% was glued together.
\KOMAoptions{BCOR = 0mm}

% I want the bibliography page to appear in the table of contents. Note
% that this entry is not numbered. Change this to bibliographynumbered to
% get it numbered.
\KOMAoptions{toc = bibliography}

% I want the table of contents to be indented the default way. If you have
% a very depply nested document structure, change this value according to
% the KOMA script documentation.
\KOMAoptions{toc = graduated}

% make tables of figures and tables appear in the table of contents
\KOMAoptions{toc = listof}

% No dot after heading numbers.
\KOMAoptions{numbers = noenddot}

% begin new chapters always on the right side
\KOMAoptions{open = right}

% Draft?
\KOMAoptions{draft = false}

% Normal sized headings
\KOMAoptions{headings = normal}

% Do not prefix chapter headings with the word "chapter"
\KOMAoptions{chapterprefix = false}

% same for appendix
\KOMAoptions{appendixprefix = true}

% calculate the DIV
\KOMAoptions{DIV = 10}